\documentclass{beamer}
\usetheme{CambridgeUS}
\title{Assignment 7}

\author{Abhishek Kumar}

\institute{IIT Hyderabad}

\date{22May,2022}
\setbeamertemplate{caption}[numbered]{}

\usepackage{enumitem}
\usepackage{tfrupee}
\usepackage{amsmath}
\usepackage{amssymb}
\usepackage{gensymb}
\usepackage{graphicx}
\usepackage{txfonts}

\def\inputGnumericTable{}

\usepackage[latin1]{inputenc}                                 
\usepackage{color}                                            
\usepackage{array}                                            
\usepackage{longtable}                                        
\usepackage{calc}                                             
\usepackage{multirow}                                         
\usepackage{hhline}                                           
\usepackage{ifthen}
\usepackage{caption} 
\captionsetup[table]{skip=3pt}  
\providecommand{\pr}[1]{\ensuremath{\Pr\left(#1\right)}}
\providecommand{\cbrak}[1]{\ensuremath{\left\{#1\right\}}}
\renewcommand{\thefigure}{\arabic{table}}
\renewcommand{\thetable}{\arabic{table}}   
\providecommand{\brak}[1]{\ensuremath{\left(#1\right)}}
\begin{document}
\begin{frame}
		\titlepage
	\end{frame}
	\begin{frame}{Outline}
		\tableofcontents
	\end{frame}
\begin{frame}{Question Statement}
\section{Question}
\textbf{Question:}Two players A and B play a series of match on a condition that A will win the series if he succeeds in winning m matches before B wins n matches.The probability of winning a match for A is $p$ and B is $q=1-p$.What is probability that A will win the series?
\end{frame}

\begin{frame}{Solution}
\section{solution}
\textbf{Solution:}Consider
\begin{table}[ht!]
\centering
			\input{table1.tex}
			\caption{}
			\label{Table:1}
\end{table}
Clearly by the end of $(m+n-1)$th match there must be a winner and $P_A+P_B=1$.Question asks to find $P_A$.
\end{frame}
\begin{frame}
  A can win in the following mutually exclusive ways.  
\begin{table}[ht!]
\centering
			\input{table2.tex}
			\caption{}
			\label{Table:2}
\end{table}
where
\begin{align}
   k=0,1,2,....,n-1
\end{align}
$X_1,X_2,...,X_n_-_1$ are mutually exclusive events.\\
if A and B are mutually exclusive events then $P(A+B)=P(A)+P(B)$.
\begin{align}
   \Rightarrow P_A=P(X_1+X_2+X_3+...+X_n_-_1)=P(X_1)+P(X_2)+..+P(X_n_-_1)
\end{align}
\end{frame}
\begin{frame}
\begin{block}{To find $P(X_k)$}
For A to win $m$ matches in exactly $m+k$ matches, A must win the last game and and $(m-1)$ matches matches in any order among the first $(m-k+1)$ matches.\\\\
    $P(X_k)=P$(A wins (m-1) matches among first (m+k-1) matches) $\times$ P(A wins the last game)
    \begin{align}
        &\Rightarrow P(X_k)=\binom{(m+k-1)}{(m-1)} \times {p}^{m-1} \times q^{k}\times p\\
        &\Rightarrow P(X_k)=((m-k+1)! {\displaystyle} \times p^{m} \times q^{k}) /(m-1)! {\displaystyle}
    \end{align}
\end{block}

\begin{align}
&P_A=\sum P(X_k) \\
&\Rightarrow P_A=p^{m}(1+(m/1) \times q+..\nonumber\\
&...+(m(m+1)..(m+n-2)/1\times2...\times(n-1))\times q^{n-1})
 \end{align}

 \end{frame}

\end{document}
