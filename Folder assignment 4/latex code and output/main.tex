\documentclass[journal, 8pt, twocolumn]{IEEEtran}

\usepackage{amsmath}
\usepackage{graphicx}
\title{Assignment 4 \\}
\author{Abhishek Kumar\\AI21BTECH11003}
\begin{document}
	\maketitle
	\textbf{Question:}If A,B,C are three events associated with a random experiment,prove that 
	\begin{align}
	&P(A+B+C)=P(A)+P(B)+P(C)-P(AB)-P(BC)\\\nonumber
	&-P(CA)+P(ABC)
	\end{align}
	\textbf{Solution:}
	Consider
	\begin{align}
	   &E=B+C \\ 
	   &P(A+B+C)=P(A+E)=P(A)+P(E)-P(AE)\\
    &P(E) = P(B+C)=P(B)+P(C)-P(BC)\\
    &AE = A(B+C) = (AB)+(AC)\\
    &P(AE)=P[(AB)+(AC)]=P(AB)+P(AC)-P[(AB)(AC)]\\
    &\Rightarrow P(AE) = P(AB)+P(AC) -P(ABC)
	\end{align}
	using equation(3) and equation(4) and equation(7)
	\begin{align}
	&P(A+B+C)=P(A)+P(B)+P(C)-P(AB)-P(BC)\\\nonumber
	&-P(CA)+P(ABC)
	\end{align}
	
	  \begin{figure}[h]
	    \centering
	    \includegraphics[scale=0.5]{figure_5.png}
	    \caption{By this figure generated by python code,we can verify equation (3) intutively}
	    \label{fig:my_label}
	\end{figure}
	
\begin{figure}[h]
	    \centering
	    \includegraphics[scale=0.5]{figure_6.png}
	    \caption{By this figure generated by python code,we can verify equation (8) intutively}
	    \label{fig:my_label}
	\end{figure}
\end{document}
