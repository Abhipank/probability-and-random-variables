\documentclass[journal,12pt,twocolumn]{IEEEtran}
\usepackage{setspace}
\usepackage{gensymb}

\usepackage{amsfonts}

\usepackage{caption}
%\usepackage{multirow}
%\usepackage{multicolumn}
%\usepackage{subcaption}
%\doublespacing
\singlespacing
\usepackage{csvsimple}
\usepackage{amsmath}
\usepackage{multicol}
%\usepackage{enumerate}
\usepackage{amssymb}
%\usepackage{graphicx}
\usepackage{newfloat}
%\usepackage{syntax}
\usepackage{listings}


\usepackage{tikz}
\usetikzlibrary{shapes,arrows}



%\usepackage{graphicx}
%\usepackage{amssymb}
%\usepackage{relsize}
%\usepackage[cmex10]{amsmath}
%\usepackage{mathtools}
%\usepackage{amsthm}
%\interdisplaylinepenalty=2500
%\savesymbol{iint}
%\usepackage{txfonts}
%\restoresymbol{TXF}{iint}
%\usepackage{wasysym}
\usepackage{amsthm}
\usepackage{mathrsfs}
\usepackage{txfonts}
\usepackage{stfloats}
\usepackage{cite}
\usepackage{cases}
\usepackage{mathtools}
\usepackage{caption}
\usepackage{enumerate}	
\usepackage{enumitem}
\usepackage{amsmath}
%\usepackage{xtab}
\usepackage{longtable}
\usepackage{multirow}
%\usepackage{algorithm}
%\usepackage{algpseudocode}
\usepackage{enumitem}
\usepackage{mathtools}
\usepackage{hyperref}
%\usepackage[framemethod=tikz]{mdframed}
\usepackage{listings}
    %\usepackage[latin1]{inputenc}                                 %%
    \usepackage{color}                                            %%
    \usepackage{array}                                            %%
    \usepackage{longtable}                                        %%
    \usepackage{calc}                                             %%
    \usepackage{multirow}                                         %%
    \usepackage{hhline}                                           %%
    \usepackage{ifthen}                                           %%
  %optionally (for landscape tables embedded in another document): %%
    \usepackage{lscape}     


\usepackage{url}
\def\UrlBreaks{\do\/\do-}


%\usepackage{stmaryrd}


%\usepackage{wasysym}
%\newcounter{MYtempeqncnt}
\DeclareMathOperator*{\Res}{Res}
%\renewcommand{\baselinestretch}{2}
\renewcommand\thesection{\arabic{section}}
\renewcommand\thesubsection{\thesection.\arabic{subsection}}
\renewcommand\thesubsubsection{\thesubsection.\arabic{subsubsection}}

\renewcommand\thesectiondis{\arabic{section}}
\renewcommand\thesubsectiondis{\thesectiondis.\arabic{subsection}}
\renewcommand\thesubsubsectiondis{\thesubsectiondis.\arabic{subsubsection}}

% correct bad hyphenation here
\hyphenation{op-tical net-works semi-conduc-tor}

%\lstset{
%language=C,
%frame=single, 
%breaklines=true
%}

%\lstset{
	%%basicstyle=\small\ttfamily\bfseries,
	%%numberstyle=\small\ttfamily,
	%language=Octave,
	%backgroundcolor=\color{white},
	%%frame=single,
	%%keywordstyle=\bfseries,
	%%breaklines=true,
	%%showstringspaces=false,
	%%xleftmargin=-10mm,
	%%aboveskip=-1mm,
	%%belowskip=0mm
%}

%\surroundwithmdframed[width=\columnwidth]{lstlisting}
\def\inputGnumericTable{}                                 %%
\lstset{
%language=C,
frame=single, 
breaklines=true,
columns=fullflexible
}
 

\begin{document}
%
\tikzstyle{block} = [rectangle, draw,
    text width=3em, text centered, minimum height=3em]
\tikzstyle{sum} = [draw, circle, node distance=3cm]
\tikzstyle{input} = [coordinate]
\tikzstyle{output} = [coordinate]
\tikzstyle{pinstyle} = [pin edge={to-,thin,black}]

\theoremstyle{definition}
\newtheorem{theorem}{Theorem}[section]
\newtheorem{problem}{Problem}
\newtheorem{proposition}{Proposition}[section]
\newtheorem{lemma}{Lemma}[section]
\newtheorem{corollary}[theorem]{Corollary}
\newtheorem{example}{Example}[section]
\newtheorem{definition}{Definition}[section]
%\newtheorem{algorithm}{Algorithm}[section]
%\newtheorem{cor}{Corollary}
\newcommand{\BEQA}{\begin{eqnarray}}
\newcommand{\EEQA}{\end{eqnarray}}
\newcommand{\define}{\stackrel{\triangle}{=}}
\bibliographystyle{IEEEtran}
%\bibliographystyle{ieeetr}
\providecommand{\nCr}[2]{\,^{#1}C_{#2}} % nCr
\providecommand{\nPr}[2]{\,^{#1}P_{#2}} % nPr
\providecommand{\mbf}{\mathbf}
\providecommand{\pr}[1]{\ensuremath{\Pr\left(#1\right)}}
\providecommand{\qfunc}[1]{\ensuremath{Q\left(#1\right)}}
\providecommand{\sbrak}[1]{\ensuremath{{}\left[#1\right]}}
\providecommand{\lsbrak}[1]{\ensuremath{{}\left[#1\right.}}
\providecommand{\rsbrak}[1]{\ensuremath{{}\left.#1\right]}}
\providecommand{\brak}[1]{\ensuremath{\left(#1\right)}}
\providecommand{\lbrak}[1]{\ensuremath{\left(#1\right.}}
\providecommand{\rbrak}[1]{\ensuremath{\left.#1\right)}}
\providecommand{\cbrak}[1]{\ensuremath{\left\{#1\right\}}}
\providecommand{\lcbrak}[1]{\ensuremath{\left\{#1\right.}}
\providecommand{\rcbrak}[1]{\ensuremath{\left.#1\right\}}}
\theoremstyle{remark}
\newtheorem{rem}{Remark}
\newcommand{\sgn}{\mathop{\mathrm{sgn}}}

\title{%Convex Optimization in Python
	\logo{
	Random Numbers
	}
}
%\title{
%	\logo{Matrix Analysis through Octave}{\begin{center}\includegraphics[scale=.24]{tlc}\end{center}}{}{HAMDSP}
%}
% paper title
% can use linebreaks \\ within to get better formatting as desired
%\title{Matrix Analysis through Octave}
%
%
% author names and IEEE memberships
% note positions of commas and nonbreaking spaces ( ~ ) LaTeX will not break
% a structure at a ~ so this keeps an author's name from being broken across
% two lines.
% use \thanks{} to gain access to the first footnote area
% a separate \thanks must be used for each paragraph as LaTeX2e's \thanks
% was not built to handle multiple paragraphs
%
\author{ Abhishek Kumar \\AI21BTECH11003}
% <-this % stops a space
%\thanks{J. Doe and J. Doe are with Anonymous University.}% <-this % stops a space
%\thanks{Manuscript received April 19, 2005; revised January 11, 2007.}}
}
% note the % following the last \IEEEmembership and also \thanks - 
% these prevent an unwanted space from occurring between the last author name
% and the end of the author line. i.e., if you had this:
% 
% \author{....lastname \thanks{...} \thanks{...} }
%                     ^------------^------------^----Do not want these spaces!
%
% a space would be appended to the last name and could cause every name on that
% line to be shifted left slightly. This is one of those "LaTeX things". For
% instance, "\textbf{A} \textbf{B}" will typeset as "A B" not "AB". To get
% "AB" then you have to do: "\textbf{A}\textbf{B}"
% \thanks is no different in this regard, so shield the last } of each \thanks
% that ends a line with a % and do not let a space in before the next \thanks.
% Spaces after \IEEEmembership other than the last one are OK (and needed) as
% you are supposed to have spaces between the names. For what it is worth,
% this is a minor point as most people would not even notice if the said evil
% space somehow managed to creep in.
% The paper headers
%\markboth{Journal of \LaTeX\ Class Files,~Vol.~6, No.~1, January~2007}%
%{Shell \MakeLowercase{\textit{et al.}}: Bare Demo of IEEEtran.cls for Journals}
% The only time the second header will appear is for the odd numbered pages
% after the title page when using the twoside option.
% 
% *** Note that you probably will NOT want to include the author's ***
% *** name in the headers of peer review papers.                   ***
% You can use \ifCLASSOPTIONpeerreview for conditional compilation here if
% you desire.
% If you want to put a publisher's ID mark on the page you can do it like
% this:
%\IEEEpubid{0000--0000/00\$00.00~\copyright~2007 IEEE}
% Remember, if you use this you must call \IEEEpubidadjcol in the second
% column for its text to clear the IEEEpubid mark.
% make the title area
\maketitle
\tableofcontents
\bigskip
\renewcommand{\thefigure}{\theenumi}
\renewcommand{\thetable}{\theenumi}
%%
\section{Uniform Random Numbers}
Let $U$ be a uniform random variable between 0 and 1.
\begin{enumerate}[label=\thesection.\arabic*
,ref=\thesection.\theenumi]
\item Generate $10^6$ samples of $U$ using a C program and save into a file called uni.dat .
\\
\textbf{solution:}Download the following files and execute the  C program.
\begin{lstlisting}
wget https://github.com/Abhipank/probability-and-random-variables/blob/main/SIMULATION%20ASSIGNMENT/codes/coeffs.h
wget https://github.com/Abhipank/probability-and-random-variables/blob/main/SIMULATION%20ASSIGNMENT/codes/exrand.c
\end{lstlisting}
%
\item
Load the uni.dat file into python and plot the empirical CDF of $U$ using the samples in uni.dat. The CDF is defined as
\begin{align}
F_{U}(x) = \pr{U \le x}
\end{align}
\\
\textbf{solution:}  The following code plots Fig. \ref{fig:uni_cdf}
\begin{lstlisting}
wget https://github.com/Abhipank/probability-and-random-variables/blob/main/SIMULATION%20ASSIGNMENT/codes/cdf_plot1.1.py
\end{lstlisting}
\begin{figure}
\centering
\includegraphics[width=\columnwidth]{uni_cdf}
\caption{The CDF of $U$}
\label{fig:uni_cdf}
\end{figure}
%
\item
Find a  theoretical expression for $F_{U}(x)$.\\
\textbf{solution:}

\begin{equation}
p_X(x) = \left\{
        \begin{array}{ll}
            1 & \quad 0 \leq  x \leq 1\\
            0 & \quad x < 0,x > 1
        \end{array}
    \right.
\end{equation}
\begin{equation}
F_U(x) =Pr(U \leq x)= \left\{
        \begin{array}{ll}
            \int_{0}^1 1dx=x & \quad 0 \leq  x \leq 1\\
            0 & \quad x < 0,x > 1
        \end{array}
    \right.  
\end{equation}


\item
The mean of $U$ is defined as
%
\begin{equation}
E\sbrak{U} = \frac{1}{N}\sum_{i=1}^{N}U_i
\end{equation}
%
and its variance as
%
\begin{equation}
\text{var}\sbrak{U} = E\sbrak{U- E\sbrak{U}}^2 
\end{equation}
Write a C program to  find the mean and variance of $U$\\
\textbf{solution:}  The following code finds mean and variance of $U$ 
\begin{lstlisting}
wget https://github.com/Abhipank/probability-and-random-variables/blob/main/SIMULATION%20ASSIGNMENT/codes/que1.4.c
\end{lstlisting}

\item Verify your result theoretically given that
\end{enumerate}
%
\begin{equation}
E\sbrak{U^k} = \int_{-\infty}^{\infty}x^kdF_{U}(x)
\end{equation}
\textbf{solution:}
\begin{equation}
E[U] = \int_{-\infty}^{\infty}x^1dF_{U}(x)
= E[U]=\int_{0}^{1}x^1p_X(x) dx \nonumber
\end{equation}
\begin{equation}
\Rightarrow E[U] =0.5 \nonumber
\end{equation}
\begin{align}
    &E[U^{2}]=\int_{-\infty}^{\infty}x^2dF_{U}(x)\nonumber\\
    &\Rightarrow E[U^{2}]=\int_{0}^{1}x^2 p_X(x) dx\nonumber\\
    &\Rightarrow E[U^{2}]=\int_{0}^{1}x^2 1 dx\nonumber\\
    &\Rightarrow E[U^{2}]=1/3\nonumber
\end{align}
\begin{align}
 &var[U]=E[U-E[U]]^{2}\\
 &\Rightarrow var[U]=E[U^{2}+E[U]^{2}-2UE[U]] \nonumber\\
 &\Rightarrow var[U]=E[U^{2}]-E[U]^{2} \nonumber\\
 &\Rightarrow var[U]=1/3 -1/4 \nonumber\\
 &\Rightarrow var[U]=1/12 \nonumber
\end{align}
\section{Central Limit Theorem}
%
\begin{enumerate}[label=\thesection.\arabic*
,ref=\thesection.\theenumi]
%
\item
Generate $10^6$ samples of the random variable
%
\begin{equation}
X = \sum_{i=1}^{12}U_i -6
\end{equation}
%
using a C program, where $U_i, i = 1,2,\dots, 12$ are  a set of independent uniform random variables between 0 and 1
and save in a file called gau.dat\\
\textbf{solution:}Download the following files and execute the  C program.
\begin{lstlisting}
wget https://github.com/Abhipank/probability-and-random-variables/blob/main/SIMULATION%20ASSIGNMENT/codes/coeffs.h
wget https://github.com/Abhipank/probability-and-random-variables/blob/main/SIMULATION%20ASSIGNMENT/codes/exrand.c
\end{lstlisting}


%
\item
Load gau.dat in python and plot the empirical CDF of $X$ using the samples in gau.dat. What properties does a CDF have?
\\
\textbf{solution} The CDF of $X$ is plotted in Fig. \ref{fig:gauss_cdf}
\begin{lstlisting}
wget https://github.com/Abhipank/probability-and-random-variables/blob/main/SIMULATION%20ASSIGNMENT/codes/que2.2.py
\end{lstlisting}
CDF is denoted by $\phi(x)$.Here are  properties for CDF of normal distribution:
\begin{enumerate}
\item
    $\lim_{x \to +\infty} \phi(x)=1$
\item
    $\lim_{x \to -\infty} \phi(x)=0$
\item
    $\phi(x)=1-\phi(-x)$
\item
    $\phi(x)$ is non decreasing.
\end{enumerate}

\begin{figure}
\centering
\includegraphics[width=\columnwidth]{gauss_cdf}
\caption{The CDF of $X$}
\label{fig:gauss_cdf}
\end{figure}
\item
Load gau.dat in python and plot the empirical PDF of $X$ using the samples in gau.dat. The PDF of $X$ is defined as
\begin{align}
p_{X}(x) = \frac{d}{dx}F_{X}(x)
\end{align}
What properties does the PDF have?
\\
\textbf{solution:} The PDF of $X$ is plotted in Fig. \ref{fig:gauss_pdf} using the code below
\begin{lstlisting}
wget https://github.com/Abhipank/probability-and-random-variables/blob/main/SIMULATION%20ASSIGNMENT/codes/pdf_plot.py
\end{lstlisting}
PDF is denoted by $f(x)$.Here are  properties for PDF of normal distribution:
\begin{enumerate}
\item
    All three mean,mode,median are the same and are under the peak of pdf curve.
\item
    $f(x)=f(-x)$.The function is symmetric about the y-axis.
\item
    Within one standard deviation, 68\% of all observations lie.
\item
    $\int_{-\infty}^{\infty}f(x)dx=1$
\item
  $\forall x \in$ $R$ f(x) $\geq$ 0
\end{enumerate}
\begin{figure}
\centering
\includegraphics[width=\columnwidth]{gauss_pdf}
\caption{The PDF of $X$}
\label{fig:gauss_pdf}
\end{figure}
\item Find the mean and variance of $X$ by writing a C program.\\
\textbf{solution:}  The following code finds mean and variance of $X$ 
\begin{lstlisting}
wget https://github.com/Abhipank/probability-and-random-variables/blob/main/SIMULATION%20ASSIGNMENT/codes/que2.4.c
\end{lstlisting}
\item Given that 
\begin{align}
p_{X}(x) = \frac{1}{\sqrt{2\pi}}\exp\brak{-\frac{x^2}{2}}, -\infty < x < \infty,
\end{align}
repeat the above exercise theoretically.
%
\end{enumerate}
\textbf{solution:}
\begin{equation}
E[U] = \int_{-\infty}^{\infty}x^1dF_{U}(x)
= E[U]=\int_{-\infty}^{\infty}x^1p_X(x) dx \nonumber
\end{equation}
\begin{align}
&\Rightarrow E[U] =\int_{-\infty}^{\infty}x^1\frac{1}{\sqrt{2\pi}}\exp\brak{-\frac{x^2}{2}} dx \nonumber
\end{align}
$\Rightarrow$ E[U] =0 because integrand is odd function.
\begin{align}
    &E[U^{2}]=\int_{-\infty}^{\infty}x^2dF_{U}(x)\nonumber\\
    &\Rightarrow E[U^{2}]=\int_{-\infty}^{\infty}x^2 p_X(x) dx\nonumber\\
    &\Rightarrow E[U^{2}]=\int_{-\infty}^{\infty}x^2\frac{1}{\sqrt{2\pi}}\exp\brak{-\frac{x^2}{2}} dx \nonumber
\end{align}
$\Rightarrow$ $E[U^{2}]=1$  by integration by parts \\and $erf(\infty)-erf(-\infty)=2=\frac{\sqrt{2}}{\sqrt(\pi)}\int_{-\infty}^{\infty}e^{-x^{2}/2}$
\begin{align}
 &var[U]=E[U-E[U]]^{2}\\
 &\Rightarrow var[U]=E[U^{2}+E[U]^{2}-2UE[U]] \nonumber\\
 &\Rightarrow var[U]=E[U^{2}]-E[U]^{2} \nonumber\\
 &\Rightarrow var[U]=1 -0 \nonumber\\
 &\Rightarrow var[U]=1 \nonumber
\end{align}
\section{From Uniform to Other}
\begin{enumerate}[label=\thesection.\arabic*
,ref=\thesection.\theenumi]
%
\item
Generate samples of 
%
\begin{equation}
V = -2\ln\brak{1-U}
\end{equation}
%
and plot its CDF.\\
\textbf{solution:}Download the following files and execute the  C program.
\begin{lstlisting}
wget https://github.com/Abhipank/probability-and-random-variables/blob/main/SIMULATION%20ASSIGNMENT/codes/coeffs.h
wget https://github.com/Abhipank/probability-and-random-variables/blob/main/SIMULATION%20ASSIGNMENT/codes/que3.1.c
\end{lstlisting}
The CDF of $V$ is plotted in Fig. \ref{fig:exp_cdf}
\begin{lstlisting}
wget https://github.com/Abhipank/probability-and-random-variables/blob/main/SIMULATION%20ASSIGNMENT/codes/que3.1.py
\end{lstlisting}
\begin{figure}
\centering
\includegraphics[width=\columnwidth]{exp_cdf}
\caption{The CDF of $V$}
\label{fig:exp_cdf}
\end{figure}

\item Find a theoretical expression for $F_V(x)$.\\
\textbf{solution:}

\begin{equation}
p_X(x) = \left\{
        \begin{array}{ll}
            \lambda e^{-\lambda x} & \quad x \geq 0\\
            0 & \quad x < 0
        \end{array}
    \right.
\end{equation}
\begin{equation}
F_U(x) =Pr(U \leq x)= \left\{
        \begin{array}{ll}
            \int_{0}^x \lambda e^{-\lambda x}dx=1 -e^{-\lambda x} & \quad x \geq 0\\
            0 & \quad x < 0
        \end{array}
    \right.  
\end{equation}

%
%\item
\end{enumerate}
\end{document}