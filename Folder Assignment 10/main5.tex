\documentclass{beamer}
\usetheme{CambridgeUS}
\title{Assignment 10}
\graphicspath{{images/}}
\author{Abhishek Kumar}

\institute{IIT Hyderabad}

\date{1 May,2022}
\newcommand{\Int}{\int\limits}

\begin{document}
	\begin{frame}
		\titlepage
	\end{frame}
	\begin{frame}{Outline}
		\tableofcontents
	\end{frame}
	\section{Question}
	\begin{frame}{Question Statement}
		
		\textbf{Question:} The random variables ${X_k}$ are independent with densities $f_k(x)$ and the random variable variable $n$ is independent of $X_k$ with $P(n=k)=p_k$.Show that
		\begin{align}
			s=\sum X_k   \Rightarrow   F_s(s)=\sum p_k[f_1(s) \times f_2(s) ...... \times f_n(s)]
		\end{align}
	\end{frame}
	\section{Solution}
	\begin{frame}{Solution}
		\textbf{Solution:}
		f denotes p.d.f. and F denotes c.d.f.
		\begin{alertblock}{Approach}
			we will use the concept of conditional probability and extension of bayes theorem in probability density distribution
		\end{alertblock}
		We shall first determine the conditional probability
		\begin{align}
			F_s(s|n=n)=P(s \leq s,n=n)/P(n=n)
		\end{align}
		The event $(s \leq s,n=n)$ consists of all outcomes such that n=n and $\sum X_k \leq s$.since the RV n is independent of RVs $X_k$,this yields 
	\end{frame}
	\begin{frame}
		\begin{align}
			F_s(s)=P(\sum X_k \leq s)P(n=n)/p(n=n)=P(\sum X_k)
		\end{align}
		From the independence of RVs $X_k$,it follows that 
		\begin{align}
			F_s(s|n=n)=f_1(s) \times f_2(s) \times f_3(s) ...... \times f_n(s)
		\end{align}
		Extending the concept of bayes theorem in probability density distribution concept,we get
		\begin{align}
			F(x)=F(x|A_1)P(A_1)+....+F(x|A_n)P(A_n)
		\end{align}
		setting $A_k=(n=k)$ in equation(5) ,we obtain
		\begin{align}
			&F_s(s)=\sum P(X_k)[f_1(s) \times f_2(s) ...... \times f_n(s)]\\
			&\Rightarrow F_s(s)=\sum P(X_k)[f_1(s) \times f_2(s) ..... \times f_n(s)]
		\end{align}
		Hence Proved
	\end{frame}
	
\end{document}