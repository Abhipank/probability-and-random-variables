\documentclass{beamer}
\usetheme{CambridgeUS}
\title{Assignment 8}

\author{Abhishek Kumar}

\institute{IIT Hyderabad}

\date{24May,2022}

\begin{document}
	\begin{frame}
		\titlepage
	\end{frame}
	\begin{frame}{Outline}
		\tableofcontents
	\end{frame}
	\section{Question}
	\begin{frame}{Question Statement}
		
		\textbf{Question:}Let $X$ represent a binomial random variable with parameters n and p.Show that and Compute
		\begin{enumerate}[(a)]
			\item $E[X]=np$
			\item $E[X(X-1)]=n(n-1)p^{2}$
			\item $E[X(X-1)(X-2)]=n(n-1)(n-2)p^{3}$
			\item $E[X^{2}]$
			\item $E[X^{3}]$
		\end{enumerate}
	\end{frame}
	\section{Solution}
	\begin{frame}{Solution}
		\textbf{Solution:}
		Let $P(X)$ be probability of random Variable X and r denotes all values that X can take.
		
		\begin{align}
			&E[X]=\sum X\times P(X=r)\\
			&\Rightarrow E[X]=\sum r \times \binom{n}{r} \times p^{r} \times (1-p)^{n-r}\\
		\end{align}
		Using Binomial coefficients property:\\
		\begin{align}
			&\binom{n}{r}=n/r \times \binom{n-1}{r-1}\\
			&\Rightarrow E[X]=\sum r \times n/r \times \binom{n-1}{r-1} \times p^{r} \times (1-p)^{n-r}
		\end{align}
		
	\end{frame}
	\begin{frame}
		\begin{align}
			&\Rightarrow E[X]=np\sum \binom{n-1}{r-1} \times p^{r-1} \times (1-p)^{((n-1)-(r-1))}\\
			&\Rightarrow E[X]=np(p+1-p)^{n-1}\\
			&\Rightarrow E[X]=np
		\end{align}
		\begin{align}
			&E[X(X-1)]=\sum X(X-1)\times P(X=r)\\
		\end{align}
		\begin{align}
			&\Rightarrow E[X(X-1)]=\sum r(r-1)\times \binom{n}{r} \times p^{r} \times (1-p)^{n-r}
		\end{align}
		
		
	\end{frame}
	\begin{frame}
		Using Binomial coefficients property:\\
		\begin{align}
			&\binom{n}{r}=n/r \times (n-1)/(r-1) \times \binom{n-1}{r-1}\\
			&\Rightarrow E[X(X-1)]=\sum r(r-1) \times n(n-1)/r(r-1) \times \binom{n-1}{r-1}p^{r}(1-p)^{n-r}
		\end{align}
		\begin{align}
			&\Rightarrow E[X(X-1)]=n(n-1)p^{2}\sum \binom{n-2}{r-2} \times p^{r-2} \times (1-p)^{((n-2)-(r-2))}\\
			&\Rightarrow E[X(X-1)]=n(n-1)p^{2}(p+1-p)^{n-2}\\
			&\Rightarrow E[X(X-1)]=n(n-1)p^{2}
		\end{align}
	\end{frame}
	\begin{frame}
		\begin{align}
			&E[X(X-1)(X-2)]=\sum X(X-1)(X-2)\times P(X=r)\\
			&\Rightarrow E[X(X-1)(X-2)]=\sum r(r-1)(r-2) \times \binom{n}{r} \times p^{r} \times (1-p)^{n-r}
		\end{align}
		
		Using Binomial coefficients property:\\
		\begin{align}
			&\binom{n}{r}=n(n-1)(n-2)/(r)(r-1)(r-2)\times \binom{n-3}{r-3}\\
			&\Rightarrow E[X(X-1)(X-2)]=n(n-1)(n-2)p^{3}\sum\binom{n-3}{r-3}p^{r-3}(1-p)^{(n-r)}\\
			&\Rightarrow E[X(X-1)(X-2)]=n(n-1)(n-2)p^{3}
		\end{align}
	\end{frame}
	\begin{frame}
		From equation 16
		\begin{align}
			&E[X(X-1)]=n(n-1)p^{2}\\
			&\Rightarrow E[X^{2}]-E(X)=n(n-1)p^{2}\\
			&\Rightarrow E[X^{2}]=n(n-1)p^{2}+np
		\end{align}
		From equation 21
		\begin{align}
			&E[X(X-1)(X-2)]=n(n-1)(n-2)p^{3}\\
			&\Rightarrow E[X^{3}]-3\times E[X^{2}]+2 \times E[X]=n(n-1)p^{2}\\
			&\Rightarrow E[X^{3}]=n(n-1)(n-2)p^{3}+3n(n-1)p^{2}+np
		\end{align}
	\end{frame}
	
	
	
\end{document}