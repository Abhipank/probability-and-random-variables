\documentclass[journal, 8pt, twocolumn]{IEEEtran}


% FOR TABLE (according to table generated by ssconvert)
\def\inputGnumericTable{}
\usepackage[latin1]{inputenc}
\usepackage{color}
\usepackage{array}
\usepackage{longtable}
\usepackage{calc}
\usepackage{multirow}
\usepackage{hhline}
\usepackage{ifthen}
\usepackage{lscape}

\usepackage{amsmath}
\usepackage{graphicx}
\title{Assignment 5 \\}
\author{Abhishek Kumar\\AI21BTECH11003}
\begin{document}
	\maketitle
	\textbf{Question:}Suppose that the reliability of HIV test is specified as follows:\\
	Of people having HIV, $90$\% of the test detect the disease but 10\% go undetected.Of people free of HIV,$99$\% of the test are judged HIV $-$ve but $1$\% are diagnosed as showing HIV $+$ve.From a large population of which only $0.1$\% have HIV,one person is selected at random,given the HIV test and pathologist reports him/her as HIV $+$ve.What is the probability that the person actually has HIV ?\\
	
	\textbf{Solution:}
	Let $E$ be the event that the person selected is actually having HIV and $A$ be the event that the person's HIV test is diagnosed as $+$ve.We need to find $P(E|A)$ i.e. probability that person is actually having HIV provided he has been diagnosed as HIV $+$ve .
	
	\begin{table}[ht!]
		\centering
		\input{table1.tex}
		\caption{Events}
		\label{Table:1}
	\end{table}
	
	
	Also    $E'$ denotes the event that person selected is actually not having HIV.
	\begin{align}
	&P_E(1)=0.1\%=0.1/100=0.001\\
	&P_E(0)=1-P_E(1)=1-0.001=0.999\\
	&P_A(1)=P_E(1)P_A_|_E(1|1)+P_E(1)P_A_|_E(1|0)\\
	&P_A_|_E(1|1)=1\%=0.9\\
	&P_A_|_E(1|0)=90\%=0.01
	\end{align}
	Using Bayes Theorem,
	\begin{align}
	    &P_E_|_A(1|1)=P_E_A(1,1)/P_A(1)=P_E(1)P_A_|_E(1|1)/(P_E(1)P_A_|_E(1|1)+P_E(0)P_A_|_E(1|0))\\
	    &\Rightarrow P_E_|_A(1|1)=0.001\times0.9/(0.001\times0.9+0.999\times0.01)\\
	    &\Rightarrow P_E_|_A(1|1)=0.083 
	\end{align}
	
\end{document}
