\documentclass[journal, 8pt, twocolumn]{IEEEtran}

\usepackage{amsmath}
\usepackage{graphicx}

% FOR TABLE (according to table generated by ssconvert)
\def\inputGnumericTable{}
\usepackage[latin1]{inputenc}
\usepackage{color}
\usepackage{array}
\usepackage{longtable}
\usepackage{calc}
\usepackage{multirow}
\usepackage{hhline}
\usepackage{ifthen}
\usepackage{lscape}


\title{Assignment 3 \\}
\author{Abhishek Kumar\\AI21BTECH11003}

\begin{document}
	
	\maketitle
	
	\textbf{Question:}
	
	The following table gives the points scored by a kabaddi team in a series of match:
	\begin{table}[!htb]
		\input{data.tex}
		\caption{}
		\label{table:data}
	\end{table}
	
	
	find median of the points scored by team.
	
	\textbf{Solution:}
	
	Firstly we arrange the different scores/points in ascending order.
	
	\begin{table}[!htb]
		\input{data1.tex}
		\caption{}
		\label{table:data1}
	\end{table}
	
	Total number of matches=$16$\\
	There are two middle terms: $(16/2)$th and $(16/2)+1$ th terms\\
	so median should be mean of $8$th(match no.$9$) and $9$th terms(match no.$7$)
	\begin{align}
		&median=(10+14)/2\\
		&\Rightarrow median=12
	\end{align}
	
	\begin{figure}
		\centering
		\includegraphics[scale=0.7]{Figure_12.png}
		\caption{Y-axis Illustrating different scores scored by a kabaddi team and X-axis shows match no. }
		\label{fig:1}
	\end{figure}
	
	
\end{document}