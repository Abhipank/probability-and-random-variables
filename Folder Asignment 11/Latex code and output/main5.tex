\documentclass{beamer}
\usetheme{CambridgeUS}
\title{Assignment 11}
\graphicspath{{images/}}
\author{Abhishek Kumar}
\usepackage{enumerate}
\institute{IIT Hyderabad}

\date{3 May,2022}


\begin{document}
	\begin{frame}
		\titlepage
	\end{frame}
	\begin{frame}{Outline}
		\tableofcontents
	\end{frame}
	\section{Question}
	\begin{frame}{Question Statement}
		
		\textbf{Question:} From past records ,it is known that the life length of tyres of type A is a random variable $X$  with standard deviation $\sigma=5000$ miles  and .We test  64 samples and find their average life length $\overline X=25000$ miles.Find the 0.9 confidence interval of the mean.
	\end{frame}
	\section{Solution}
	\begin{frame}{Solution}
		\textbf{Solution:}
		
		\begin{alertblock}{Approach}
			Two things affect the width of any confidence interval :
			\begin{enumerate}
				\item Variation in the population
				\item Sample size
			\end{enumerate}
		\end{alertblock}
		Central limit theorem underpins following:\\
		Confidence interval$=$
		\begin{align}
			\overline X \pm t \times \sigma/ \sqrt{n}
		\end{align}
		where $t=$test statistic,$n=$sample size
	\end{frame}
	\begin{frame}
		t-value is calculated from sample data during Hypothesis tests and confidence level.From a standard chart ,t-value=1.6448. 
		\begin{align}
			&\sigma =5000\\
			&\overline X=25000\\
			&n=64\\
			&confidence=90\%
		\end{align}
		Plugging the values in the equation(1),we get:\\
		Confidence interval$=$
		\begin{align}
			&25000 \pm 1028\nonumber\\
			&\Rightarrow [23972,26028]\nonumber
		\end{align}
		\begin{alertblock}{Result}
			We can say with 90\% confidence that most values for tyre life lies between $[23972,26028]$ miles.
		\end{alertblock}
	\end{frame}
	
\end{document}