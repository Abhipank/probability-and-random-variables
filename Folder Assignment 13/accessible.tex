\documentclass{beamer}
\usetheme{CambridgeUS}
\title{Assignment 13}
\graphicspath{{images/}}
\author{Abhishek Kumar}
\usepackage{enumerate}
\institute{IIT Hyderabad}

\date{18 May,2022}


\begin{document}
	\begin{frame}
		\titlepage
	\end{frame}
	\begin{frame}{Outline}
		\tableofcontents
	\end{frame}
	\section{Question}
	\begin{frame}{Question Statement}
		
		\textbf{Question:} if a state $e_j$ is accessible from a persistent state $e_i$,then $e_i$ is also accessible from $e_j$ and moreover $e_j$ is persistent.
	\end{frame}
	\section{Solution}
	\begin{frame}{Solution}
		\textbf{Solution:}
		suppose a state $e_j$ is accessible from a persistent  state $e_i$,but $e_i$ is not accessible from $e_j$.Thus system goes from $e_i$ to $e_j$ in a certain number of steps with positive probability $p_i_j^{m}=a$ and after that it does not return to $e_i$.consequently starting from  $e_i$ the probability of the system not returning to $e_i$ is at least a or the probability of the system eventually returning to $e_i$ cannot exceed $1-a$.Thus $f_1_1 \le 1-a$  .But $1-a$ is strictly less than 1,contradicting the assumption that $e_i$ is persistent .Hence $e_i$ must be accessible from $e_j$,that is,$p_j_i^{r}=b>0$ for some r.\\\\
		
		from above ,we have 
		
		\begin{align}
			&p_i_j^{n+m} \ge p_i_k^{m}p_k_j^{n} \\
			&\Rightarrow p_i_j^{n+m+r} \ge p_i_j^{m}p_j_i^{n+r} \ge p_i_j^{m} p_j_j^{n}p_j_i^{r} = abp_i_j^{n}
		\end{align}
		Similarly,
		\begin{align}
			p_j_j^{n+m+r} \ge p_j_i^{r}p_i_i^{n}p_i_j^{m}=abp_i_i^{n}
		\end{align}
		
	\end{frame}
	\begin{frame}
		
		\begin{alertblock}{Result}
			Thus the two series $\sum$ $p_i_i^{n}$ and $\sum$ $p_j_j^{n}$ converge or diverge together .But $\sum$ p_i_i^{n} = $\infty$ ,since $e_i$ is persistent and it follows now that $e_j$ is also persistent .This completes the proof.
		\end{alertblock}
	\end{frame}
	
\end{document}